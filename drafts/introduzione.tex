\section{Introduzione}
\subsection{Guida remota e guida autonoma}
La guida autonoma rappresenta un complesso sistema tecnologico che integra una serie di avanzate tecnologie, metodologie e tecniche finalizzate a consentire il movimento di un veicolo senza necessità di intervento umano diretto. Un veicolo autonomo è infatti dotato della capacità di analizzare l'ambiente circostante, elaborare un percorso ottimale in base ai dati raccolti, e seguire tale percorso in modo autonomo. Questi processi fondamentali vengono generalmente suddivisi in tre fasi distinte: percezione (perception), pianificazione (planning) e controllo (control).
La percezione riguarda la capacità del veicolo di raccogliere informazioni dall'ambiente circostante attraverso sensori avanzati, che possono includere telecamere, radar, lidar, e altre tecnologie di rilevamento. Questi dati vengono poi elaborati nella fase di pianificazione, durante la quale il sistema valuta le possibili traiettorie e sceglie il percorso più sicuro ed efficiente da seguire. Infine, la fase di controllo si occupa dell'esecuzione del movimento del veicolo lungo il percorso stabilito, garantendo che vengano seguite le decisioni prese nella fase di pianificazione.
D'altro canto, il concetto di guida remota si riferisce a un tipo di guida in cui le decisioni relative alla direzione e al movimento del veicolo vengono prese da un essere umano, che opera a distanza utilizzando tecnologie quali sensori, attuatori, e le reti di comunicazione. In questo scenario, l'essere umano non si trova fisicamente all'interno del veicolo, ma interagisce con esso attraverso un'interfaccia remota, sfruttando la trasmissione dei dati in tempo reale per monitorare e controllare il veicolo. Tale approccio combina l'intelligenza umana con l'automazione tecnologica, rendendo possibile la guida di veicoli in situazioni in cui la presenza fisica del conducente potrebbe non essere necessaria o praticabile.
\subsection{Scopo della tesi}
L'obiettivo principale di questa tesi è sviluppare un veicolo capace di operare in modo autonomo in condizioni di guida normali, sfruttando un avanzato sistema di guida autonoma, ma che possa anche, su richiesta, passare alla modalità di guida remota. Questo approccio duale consente al veicolo di navigare in modo completamente indipendente quando le circostanze lo permettono, utilizzando tecnologie di percezione, pianificazione e controllo integrate, ma offre al contempo la flessibilità di essere controllato a distanza da un operatore umano qualora la situazione lo richieda. La possibilità di commutare tra guida autonoma e remota mira a garantire la massima sicurezza, adattabilità e versatilità del veicolo in una varietà di scenari operativi.
