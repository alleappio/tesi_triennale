\section{Funzionamento}
Nella seguente sezione si descrive il funzionamento generale dello stack di guida remota e come un server si può interfacciare con il veicolo per svolgere tali operazioni
\subsection{Informazioni scambiate tra veicolo e server}
Per adempire allo scopo di guida remota sono necessarie due figure distinte: Il veicolo ed un server. Il veicolo è stato descritto nella sezione precedente, il server invece è quella figura del sistema con cui l'umano si interfaccia e che fornisce sia i dati prelevati dal veicolo sia un metodo per il controllo del veicolo.
Per garantire il funzionamento della struttura è quindi necessario lo scambio di informazioni tra le due parti. Queste informazioni sono:

\begin{itemize}
  \item I messsaggi di controllo per i motori
  \item La pointcloud fornita dal lidar
  \item l'odometria calcolata dal veicolo
\end{itemize}

Le informazioni tra il server ed il veicolo sono scambiate tramite protocollo di rete MQTT, un protocollo studiato per il mondo dell'IOT e per tali applicazioni.
Sia a bordo del veicolo che sul server è istanziata un versione di ROS (Robotic Operative System) ovvero un software utilizzato per creare diversi nodi (o processi paralleli) e per permettere lo scambio di informazioni tra essi (Tramite topic e messaggi).
\subsection{Funzionamento lato veicolo}
A bordo del mezzo sono istanziati due nodi ROS: uno incaricato di gestire l'invio dei dati del sensore e dell'odometria, ed uno incaricato di ricevere i messaggi di controllo. Per comodità li chiameremo rispettivamente \textit{telemetry node} e \textit{control node}. 
Come descritto prima il \textit{telemetry node} si incarica d ricevere messaggi da ROS contenenti la pointcloud del lidar e l'odometria del mezzo, per poi formattare tali messaggi come stringhe JSON ed inviare questi messaggi tramite protocollo MQTT al server in due topic dedicati.
Per quanto riguarda il \textit{control node} invece, questo si incarica di ricevere messaggi dal server che riguardano il controllo del mezzo, una volta ricevuti i messaggi come stringa JSON questi vengono convertiti in messaggi ROS e vengono inviati ai driver del rovere che permettono poi il controllo tramite CAN del veicolo.
\subsection{Funzionamento lato server}
\noindent Il server, a differenza del veicolo, esegue un solo nodo ros. Lo scopo di tale nodo non è tanto quello di prendere decisioni sul movimento del veicolo ma è quello di ricevere i dati dal veicolo, farli processare da un terzo ente, ricevere le decisioni di questo terzo ente espresse come azioni che il veicolo deve eseguire ed inviarle al veicolo.
\noindent Nello specifico tale nodo riceve tramite protocollo MQTT i dati dal veicolo, per poi formattarli come messaggi ros e pubblicarli sui relativi topic ros. Una volta che questi dati vengono ripubblicati questi possono essere elaborati o da uno stack di guida autonoma o da un effettivo pilota che a distanza vedrà i dati rilevati dal veicolo e prenderà decisioni sul controllo.
\noindent Una volta prese queste decisioni dal terzo ente, queste verranno pubblicate sul topic ros addetto al trasporto di tali informazioni e a cui il nodo ros del server è iscritto. Una volta che che il nodo avrà letto il dato, lo formatterà in una stringa JSON e la invierà tramite MQTT al veicolo che eseguirà tali comandi
