\section{Funzionamento}
Nella seguente sezione si descrive il funzionamento generale dello stack di guida remota e come un server si può interfacciare con il veicolo per svolgere tali operazioni
\subsection{Informazioni scambiate tra veicolo e server}
Per adempire allo scopo di guida remota sono necessarie due figure distinte: Il veicolo ed un server. Il veicolo è stato descritto nella sezione precedente, il server invece è quella figura del sistema con cui l'umano si interfaccia e che fornisce sia i dati prelevati dal veicolo sia un metodo per il controllo del veicolo.
Per garantire il funzionamento della struttura è quindi necessario lo scambio di informazioni tra le due parti. Queste informazioni sono:
- I messsaggi di controllo per i motori
- La pointcloud fornita dal lidar
- l'odometria calcolata dal veicolo
Le informazioni tra il server ed il veicolo sono scambiate tramite protocollo di rete MQTT, un protocollo studiato per il mondo dell'IOT e per tali applicazioni.
Sia a bordo del veicolo che sul server è istanziata un versione di ROS (Robotic Operative System) ovvero un software utilizzato per creare diversi nodi (o processi paralleli) e per permettere lo scambio di informazioni tra essi (Tramite topic e messaggi).
\subsection{Funzionamento lato veicolo}
A bordo del mezzo sono istanziati due nodi ROS: uno incaricato di gestire l'invio dei dati del sensore e dell'odometria, ed uno incaricato di ricevere i messaggi di controllo. Per comodità li chiameremo rispettivamente \textit{telemetry node} e \textit{control node}. 
Come descritto prima il \textit{telemetry node} si incarica d ricevere messaggi da ROS contenenti la pointcloud del lidar e l'odometria del mezzo, per poi formattare tali messaggi come stringhe JSON ed inviare questi messaggi tramite protocollo MQTT al server in due topic dedicati.
Per quanto riguarda il \textit{control node} invece, questo si incarica di ricevere messaggi dal server che riguardano il controllo del mezzo, una volta ricevuti i messaggi come stringa JSON questi vengono convertiti in messaggi ROS e vengono inviati ai driver del rovere che permettono poi il controllo tramite CAN del veicolo.
\subsection{Funzionamento lato server}
