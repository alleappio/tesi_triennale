\section{Funzionamento}
Nella seguente sezione si descrive il funzionamento del veicolo, degli algoritmi utilizzati e delle scelte progettuali.

\subsection{Stack di guida autonoma}
La prima cosa da analizzare è il funzionamento dello stack di guida autonoma.
Lo stack funziona grazie a diversi processi, divisi (come descritto nell'introduzione) in perception, planning e control, di seguito una breve censita dei nodi ROS che permettono tale funzionamento:
Per quanto riguarda la parte di perception, vediamo due aspetti, i sensori e l'algoritmo di localizzazione:

\begin{itemize}
  \item \textbf{urg\_node}: È il nodo che permette di pubblicare sul topic ROS \textit{/scan} le pointcloud rilevate dal sensore Lidar, il tipo di dato utilizzato è chiamato \textit{LaserScan} e fornisce una serie di distanze che vanno insieme a formare ciò che il sensore Lidar rileva.
  \item \textbf{hunter\_ros2\_node}: Fornisce un interfaccia con il robot stesso, da questo nodo possiamo ricevere l'odometria calcolata a partire dal movimento delle ruote e, come vedremo successivamente, potremo pubblicare i comandi che il robot dovrà svolgere. Quello che interessa a noi al momento della perception è l'odometria del mezzo, che viene pubblicata sul tpoic \textit{/odometry} sottoforma di dato \textit{Odomentry}. 
  \item \textbf{particle\_filter}: Implementa l'algoritmo di localizzazione chiamato particle filter, questo algoritmo sfrutta per il suo funzionamento la mappa dell'ambiente in cui il robot si sta muovendo, l'odometria del veicolo e la pointcloud del sensore Lidar. Questo nodo pubblica la posizione calcolata sul topic ROS \textit{/pf/position} sottoforma di dato \textit{Odometry}.
\end{itemize}

Passando invece al momento del planning ci si avvale di due nodi:

\begin{itemize}
  \item \textbf{path\_logger}: Permette la registrazione di un percorso quando il veicolo viene guidato manualmentequesto percorso viene poi salvato in un file apposito.
  \item \textbf{path\_logger}: Questo nodo si occupa di pubblicare un percorso preregistrato o precalcolato da seguire, il dato è pubblicato sul topic \textit{/path}.  
\end{itemize}

Andiamo infine a descrivere il funzionamento della parte di controllo, questa è infatti composta da due nodi:

\begin{itemize}
  \item \textbf{purepursuit}: Si occupa di ricevere il percorso pubblicato sul topic \textit{/path} e a partire dalla posizione pubblicata dal nodo \textbf{particle\_filter} calcola i comandi da impartire al robot. I comandi vengono pubblicati sul topic\textit{/drive\_parameters} e sono di tipo \textit{Ackermann Stamped}, questo non è altro che un semplice tipo di messaggio ROS che incapsula il timestamp, un angolo di sterzo ed una velocità.
  \item \textbf{hunter\_ros2\_node}: Come descritto prima, questo nodo oltre a fornire l'odometria del mezzo, è anche capace di ricevere i comandi da impartire al robot. Il nodo è infatti in perenne ascolto sul topic \textit{/drive\_parameters} e ad ogni messaggio non farà altro che comunicare con l'interfaccia CAN del veicolo comunicandogli la velocità e l'angolo di sterzo da impostare.
\end{itemize}
