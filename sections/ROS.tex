\section{ROS}
In questa sezione si passa alla descrizione di ROS e del suo utilizzo.
\subsection{Nodi}
ROS o Robotic Operating System è un insieme di librerie e strumenti utili alla creazione di applicativi dedicati al controllo di robot.
Nello specifico ROS ci permette di creare unità di esecuzione o processi chiamati nodi. Un nodo ha la capacità di eseguire calcoli, interfacciarsi con periferiche o altro, ma la principale caratteristica di un nodo è la sua capacità di comunicare con gli altri nodi ROS in esecuzione.
\noindent Ciò ci permette di rappresentare con ogni nodo un modulo funzionale all'esecuzione della task del robot, per fare esempi pratici lo stack utilizzato per il controllo autonomo del rover è composto da diversi nodi, i principali sono: 
\noindent \begin{itemize}
  \item hunter\_ros2\_node: Gestisce la comunicazione tra i nodi ROS e l'interfaccia CAN del veicolo
  \item urg\_node: Comunica agli altri nodi la scan effettuata dal sensore lidar
  \item particle\_filter: calcola la localizzazione del mezzo a partire dalla mappa dell'ambiente e dalla scan del sensore
  \item telemetry\_node e control\_node: come descritto prima, gestiscono la comunicazione tra ROS ed MQTT 
\end{itemize}
Tutti questi nodi sono capaci di comunicano tra loro per scambiarsi informazioni utili all'esecuzione del robot.
\subsection{Comunicazione tra nodi}
Sorge però spontaneo domandarsi come questi nodi comunichino tra loro e come facciamo soprattutto a riconoscere di che tipo di informazione si tratti.
Una comunicazione ROS è formata da 3 elementi:
\begin{itemize}
  \item Topic: Il protocollo utilizzato da ROS è di tipo publish/subscribe, ciò vuol dire che durante l'esecuzione dei nodi si vanno a creare dei topic, ovvero stringhe che utilizzano come separatore il carattere '/' e che ci permettono di suddividere tutti i diversi dati da inviare. Un esempio sono le scan lidar che vengono pubblicate sul topic "/scan". Ogni nodo può decidere se fare la subscribe a quel nodo (ovvero ricevere tutti i dati inviati attraverso esso), fare delle publish (ovvero pubblicare dati su di esso) o se semplicemente ignorarlo.
  \item Message type: Una volta scelto un topic però si dovrà anche decidere quali informazioni saranno ammesse su questo, ROS fornisce diversi tipi di dato inviabile su un singolo topic. Un esempio è il tipo di dato utilizzato dal particle filter ovvero "Odometry messages", che descrivono la posizione (o meglio l'odometria) di un oggetto nello spazio ed è strutturanto nel seguente modo:

    \begin{forest}
      %for tree={grow'=90, circle, draw, l sep=5pt}
      for tree={draw}
      [Odometry
        [Header
          [timestamp]
        ]
        [Pose\_with\_covariance
          [Pose
            [Position
              [x]
              [y]
              [z]
            ]
            [Orientation
              [x]
              [y]
              [z]
              [w]
            ]
          ]
          [Covariance]
        ]
        [twist\_with\_covariance
          [Twist
            [Linear\_velocity
              [x]
              [y]
              [z]
            ]
            [Angular\_velocity
              [x]
              [y]
              [z]
            ]
          ]
          [Covariance]
        ]
      ]
    \end{forest}

  tutti i tipi di messaggio sono consultabili online sulla documentazione di ROS 
  \item Content: è il dato che dobbiamo inviare e che deve essere incapsulato nel tipo di dato fornitoci da ROS
\end{itemize}
