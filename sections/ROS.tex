\subsection{ROS} \label{ROS}
%In questa sezione si passa alla descrizione di ROS, al suo funzionamento e al suo utilizzo.

%\subsubsection{Descrizione}
Il Robotic Operating System (ROS) rappresenta un framework software di riferimento nel panorama della robotica. Si tratta di una suite di librerie e strumenti open-source progettata per semplificare e accelerare lo sviluppo di applicazioni complesse destinate a sistemi robotici di varia natura. ROS offre un'infrastruttura flessibile e modulare che consente agli sviluppatori di creare, testare e distribuire software in modo efficiente.

\noindent La scelta della versione di ROS è strettamente correlata alla distribuzione Linux utilizzata. In ambito robotico, Ubuntu è il sistema operativo più diffuso grazie alla sua stabilità e al supporto a lungo termine. Nel contesto del presente lavoro è stata selezionata la distribuzione Ubuntu 20.04 Focal Fossa, in quanto rappresenta l'ultima versione LTS (Long Term Support) compatibile con la scheda grafica GPGPU adottata come computer di bordo.

\noindent In linea con questa scelta, è stata adottata la versione ROS2 Foxy Fitzroy che, rispetto alla precedente generazione ROS1 (Noetic Ninjemys), introduce una serie di miglioramenti significativi, tra cui:

\begin{itemize}
  \item Supporto multi-linguaggio: ROS2 offre un supporto più ampio per diversi linguaggi di programmazione, tra cui C++ e Python. Tuttavia, nel caso specifico di questo progetto, si è optato per l'utilizzo esclusivo del C++ al fine di ottenere un controllo più granulare sul codice e sfruttare le elevate prestazioni offerte da questo linguaggio
  \item Modernizzazione delle librerie: le librerie di ROS2 sono state aggiornate e ampliate, offrendo funzionalità più avanzate e una maggiore integrazione con altri strumenti software
  \item Architettura distribuita e scalabile: ROS2 è progettato per gestire sistemi robotici complessi e distribuiti, con una maggiore attenzione alla scalabilità e alla resilienza
\end{itemize}

\noindent Nel corso di questa tesi l'implementazione del codice sarà interamente svolta in linguaggio C++. Questa decisione è stata guidata da diversi fattori:

\begin{itemize}
  \item Controllo a basso livello: il C++ consente un controllo più preciso sulle risorse hardware e software del sistema, permettendo di ottimizzare le prestazioni e di implementare algoritmi complessi in modo efficiente
  \item Prestazioni elevate: le elevate prestazioni che il C++ può fornire sono fondamentali in applicazioni robotiche real-time dove la latenza e la capacità di risposta del sistema sono critiche
  \item Ampia diffusione e supporto: data la sua vicinanza all'hardware, il C++ è il linguaggio di programmazione più utilizzato nell'ambito della robotica, in quanto permette di avere molteplici librerie e strumenti a disposizione, oltre che ad una maggiore facilità nella ricerca di soluzioni online a eventuali problemi
\end{itemize}

\noindent La suite ROS, oltre a fornire le componenti software di base, offre un ampio ventaglio di strumenti per lo sviluppo e il debugging. Tra questi, ROS-Visualization (RViz) riveste un ruolo cruciale, fornendo una potente interfaccia di visualizzazione tridimensionale. Progettato specificatamente per operare all'interno dell'ecosistema ROS, RViz consente agli sviluppatori di visualizzare in modo intuitivo e dettagliato i dati provenienti dai sensori e dagli attuatori del robot. Durante la fase di sviluppo, RViz è stato utilizzato in modo estensivo per la visualizzazione dei dati di telemetria, facilitando la comprensione del funzionamento del sistema e l'individuazione di eventuali anomalie. La sua capacità di rappresentare in modo grafico le informazioni provenienti da diversi nodi del sistema ha reso RViz uno strumento indispensabile per la messa a punto e la validazione degli algoritmi di localizzazione.

\subsubsection{Nodi}
\noindent Uno degli aspetti fondamentali di ROS è la sua architettura basata su nodi, che rappresentano la singola unità di esecuzione all'interno del sistema. Un nodo può essere responsabile di una vasta gamma di funzioni, tra cui eseguire calcoli, interfacciarsi con dispositivi hardware, raccogliere dati dai sensori, e molto altro. Tuttavia, la caratteristica più distintiva di un nodo ROS è la sua capacità di comunicare in maniera integrata con altri nodi attraverso un sistema di messaggistica distribuita. Questo sistema consente ai nodi di scambiare informazioni in tempo reale, permettendo una coordinazione precisa e affidabile tra i diversi componenti di un robot.

\noindent Questa struttura modulare e comunicativa rende possibile la rappresentazione di ciascuna funzione operativa del robot come un nodo distinto, favorendo una chiara separazione delle responsabilità e una maggiore facilità di sviluppo e manutenzione. Ad esempio, lo stack software utilizzato per il controllo autonomo del rover all'interno di questo progetto è costituito da una serie di nodi ROS, ognuno dei quali svolge un ruolo specifico e critico nel funzionamento complessivo del sistema.

\noindent Si può dunque concepire lo stack come un sistema in cui ciascun nodo è deputato a una specifica funzione. Ad esempio, un nodo gestisce esclusivamente la localizzazione, un altro è interamente dedicato alla trasmissione dei dati da ROS all'interfaccia di controllo del veicolo, e un terzo calcola la traiettoria. Questa suddivisione modulare consente una chiara separazione delle responsabilità, favorendo una maggiore manutenibilità e flessibilità del sistema e mentre ciascun nodo è responsabile di un sottoinsieme ben definito delle funzionalità, essi sono interconnessi e si scambiano attivamente le informazioni necessarie per coordinare le loro azioni.

\subsubsection{Comunicazione tra nodi}
La comunicazione tra nodi ROS è basata su 3 elementi fondamentali:
\begin{itemize}
  \item Topic: il protocollo utilizzato da ROS è di tipo publish/subscribe, il quale prevede che durante l'esecuzione dei nodi si vadano a creare dei topic, ovvero stringhe che utilizzano come separatore il carattere '/' e che permettono di smistare i diversi dati da inviare e/o ricevere. Un esempio sono le scan lidar che vengono pubblicate sul topic "/scan". Ogni nodo può decidere sottoscriversi (subscribe) ad un certo topic, ovvero ricevere tutti i dati inviati attraverso esso, pubblicare dati (publish), ovvero pubblicare dati su di esso, o se semplicemente ignorarlo
  
  \item Message type: una volta scelto il topic è necessario stabilire la tipologia di informazioni che verranno trasmesse attraverso lo stesso. ROS fornisce diversi tipi di dato pubblicabile su un singolo topic. Un esempio è il tipo di dato utilizzato dal particle filter e dall'odometria del mezzo, ovvero "Odometry messages", che descrivono la posizione e il movimento (o meglio l'odometria) di un oggetto nello spazio. Il messaggio specifico per l'odometria fornito da ROS è strutturanto nel seguente modo:
    \begin{figure}[H]
      \centering
      \begin{forest}
        %for tree={grow'=90, circle, draw, l sep=5pt}
        for tree={draw}
        [Odometry
          [Header
            [timestamp]
          ]
          [Pose\_with\_covariance
            [Pose
              [Position
                [x]
                [y]
                [z]
              ]
              [Orientation
                [x]
                [y]
                [z]
                [w]
              ]
            ]
            [Covariance]
          ]
          [twist\_with\_covariance
            [Twist
              [Linear\_velocity
                [x]
                [y]
                [z]
              ]
              [Angular\_velocity
                [x]
                [y]
                [z]
              ]
            ]
            [Covariance]
          ]
        ]
      \end{forest}
      \caption{Struttura del messaggio ROS "Odometry"}
    \end{figure}

  \noindent In questo particolare messaggio, come si può notare, sono contenuti: la posa, ovvero la posizione e l'orientamento del veicolo rispetto alla posa di partenza ed il Twist, ovvero la velocità lineare e quella angolare dell'oggetto al momento della misura.
  \noindent Un altro esempio di messaggio ROS è l'ackermann message, che fornisce come dati principali una velocità ed un angolo di sterzo e viene utilizzato per comunicare al robot il movimento da compiere. Tutti i tipi di messaggio sono consultabili online sulla documentazione di ROS 
  \item Content: è il dato che dobbiamo inviare e che deve essere incapsulato nel tipo di dato fornitoci da ROS
\end{itemize}

