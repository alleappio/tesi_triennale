\section{Introduzione}
\subsection{Guida remota e guida autonoma}
Con guida autonoma si intende un insieme di tecnologie, metodi e tecniche che permettono ad un veicolo di muoversi senza l'intervento umano. Un veicolo autonomo è infatti capace di esaminare l'ambiente, calcolare un percorso da seguire in base a quanto esaminato e seguire tale percorso. Generalmente questi 3 step vengono chiamati: perception, planning e control.
Quando parliamo di guida remota invece intendiamo un tipo di guida in cui, chi effettivamente prende decisioni su come il veicolo deve muoversi e su che percorso esso debba seguire, è un'essere umano, ma esso fa utilizzo di tecnologie come sensori, attuatori e la rete in generale per la guida.  
\subsection{Scopo della tesi}
Lo scopo di questa tesi è di sviluppare un veicolo ibrido che sia capace di avvalersi della guida autonoma in condizioni normali, ma che sia anche capace di essere guidato da un essere umano a distanza al momento del bisogno. 
