\section{Introduzione}
\subsection{Guida remota e guida autonoma}
La guida autonoma rappresenta un complesso sistema tecnologico che integra una serie di avanzate tecnologie, metodologie e tecniche finalizzate a consentire il movimento di un veicolo senza necessità di intervento umano diretto. Un veicolo autonomo è infatti dotato della capacità di analizzare l'ambiente circostante, elaborare un percorso ottimale in base ai dati raccolti, e seguire tale percorso in modo autonomo. Questi processi fondamentali vengono generalmente suddivisi in tre fasi distinte: percezione (perception), pianificazione (planning) e controllo (control).
La percezione riguarda la capacità del veicolo di raccogliere informazioni dall'ambiente circostante attraverso sensori avanzati, che possono includere telecamere, radar, Lidar, e altre tecnologie di rilevamento. Questi dati vengono poi elaborati nella fase di pianificazione, durante la quale il sistema valuta le possibili traiettorie e sceglie il percorso più sicuro ed efficiente da seguire. Infine, la fase di controllo si occupa dell'esecuzione del movimento del veicolo lungo il percorso stabilito, garantendo che vengano seguite le decisioni prese nella fase di pianificazione.

\noindent D'altro canto, il concetto di guida remota si riferisce a un tipo di guida in cui le decisioni relative alla direzione e al movimento del veicolo vengono prese da un essere umano, che opera a distanza utilizzando tecnologie quali sensori, attuatori, e le reti di comunicazione. In questo scenario, l'essere umano non si trova fisicamente all'interno del veicolo, ma interagisce con esso attraverso un'interfaccia remota, sfruttando la trasmissione dei dati in tempo reale per monitorare e controllare il veicolo. Tale approccio combina l'intelligenza umana con l'automazione tecnologica, rendendo possibile la guida di veicoli in situazioni in cui la presenza fisica del conducente potrebbe non essere necessaria o praticabile.

\subsection{Tecnologie IOT}
L'avvento dell'Internet delle Cose (IoT) ha inaugurato una nuova era tecnologica, caratterizzata dalla connettività pervasiva e dall'intelligenza distribuita. Questa rivoluzione digitale sta trasformando profondamente numerosi settori, tra cui la logistica. La crescente complessità delle catene di approvvigionamento globali, unita alla crescente domanda di efficienza e tracciabilità, rende l'IoT una tecnologia sempre più strategica per le aziende del settore.

\noindent L'Internet of Things (IoT) è un concetto che descrive una rete di dispositivi fisici connessi tra loro attraverso Internet, capaci di scambiare dati e comunicare con altri dispositivi, sistemi e servizi. Questi dispositivi, che possono includere sensori, elettrodomestici, veicoli (come nel nostro caso), sistemi di sicurezza, macchinari industriali e molto altro, sono dotati di sensori, software e altre tecnologie che permettono loro di raccogliere e condividere dati in tempo reale.

\noindent La guida remota, ovvero la possibilità di controllare un veicolo a distanza attraverso una connessione di rete, rappresenta una delle applicazioni più promettenti dell'IoT nel settore dei trasporti.

\subsection{Scopo della tesi}
L'obiettivo principale di questa tesi è sviluppare un veicolo capace di operare in modo autonomo in condizioni di guida normali, sfruttando un avanzato sistema di guida autonoma, ma che possa anche, su richiesta, passare alla modalità di guida remota.

\noindent Con condizioni normali si intende condizioni in cui il veicolo e l'hardware a bordo siano intatti e in cui il guidatore (se presente a bordo) sia in grado di condurre il veicolo senza problemi. Per fare un esempio possiamo immaginarci un semplice scenario che comprende un veicolo a guida autonoma con un conducente a bordo, in questi casi ci si aspetta che il conducente sia sempre vigile e che controlli il comportamento del mezzo, così che il mezzo in caso di errore o malfunzionamento, possa essere condotto dal guidatore al bisogno. Supponiamo ora invece che il guidatore non sia nelle sue condizione ottimali condizione causata da un malore o altro, in questo particolare caso il guidatore non può rimanere vigile e controllare il veicolo autonomo. In questo esempio, dunque, un operatore potrà collegarsi da remoto al veicolo per poterlo condurre all'ospedale più vicino o comunque in una zona sicura.

\noindent Questo approccio duale consente al veicolo di navigare in modo completamente indipendente quando le circostanze lo permettono, utilizzando tecnologie di percezione, pianificazione e controllo integrate, ma offre al contempo la flessibilità di essere controllato a distanza da un operatore umano qualora la situazione lo richieda. La possibilità di commutare tra guida autonoma e remota mira a garantire la massima sicurezza, adattabilità e versatilità del veicolo in una varietà di scenari operativi.

