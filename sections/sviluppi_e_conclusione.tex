\chapter{Sviluppi futuri e conclusioni}
In questo capitolo si vanno ad analizzare quali sono state le principali complicazioni che si sono riscontrate durante lo sviluppo, quali sono i punti da migliorare, eventuali soluzioni, sviluppi futuri e applicazioni pratiche della tesi.

\section{Problemi ed eventuali soluzioni}
Il primo dubbio riguardo il progetto descritto in questa tesi è sicuramente quello della latenza di rete: applicazioni come quella della guida autonoma richiedono l'esecuzione real time, ovvero che l'esecuzione di ogni singolo pezzo dello stack segue tempi predefiniti (chiamati deadline) anche nel caso peggiore.  

\noindent Sorge spontanea una riflessione circa l'impatto dell'utilizzo della rete sulla puntualità dell'esecuzione delle attività. L'integrazione di reti di comunicazione in applicazioni real-time, infatti, introduce un elemento di incertezza non trascurabile, in quanto la qualità della connessione può variare significativamente, compromettendo il rispetto delle tempistiche previste. Anche in condizioni di rete ottimali, l'introduzione di protocolli di comunicazione, seppur efficienti come MQTT, comporta un overhead di sistema che può influire negativamente sulla latenza e, di conseguenza, sulla tempestività delle operazioni.

\noindent È possibile mitigare, seppur parzialmente, le problematiche legate alla variabilità della latenza di rete attraverso l'implementazione di meccanismi di Quality of Service (QoS). Riducendo il livello di QoS, è possibile diminuire l'overhead associato ai controlli di protocollo e conseguentemente ottimizzare le prestazioni in termini di latenza. Tuttavia, questa scelta comporta un compromesso significativo in termini di affidabilità, poiché la riduzione dei controlli aumenta la probabilità di perdita di pacchetti. Analogamente, la selezione del protocollo di trasporto incide in modo determinante sulla latenza. Mentre il protocollo UDP, caratterizzato da un overhead ridotto, offre latenze inferiori, il protocollo TCP, garantendo la consegna affidabile dei dati, introduce un overhead maggiore e può comportare latenze più elevate. La scelta ottimale tra UDP e TCP dipende dal bilanciamento tra i requisiti di performance e affidabilità dell'applicazione.

\section{Sviluppi futuri}
Sebbene il progetto abbia raggiunto gli obiettivi prefissati, è evidente la necessità di integrare ulteriori funzionalità per consentirne l'utilizzo in contesti applicativi reali. 

\noindent In particolare, l'implementazione di un sistema di sicurezza informatica robusto rappresenta una priorità assoluta. Tale sistema dovrà garantire la protezione dei dati del mezzo mediante l'adozione di tecniche di crittografia e offuscamento, al fine di mitigare i rischi connessi alla trasmissione di informazioni sensibili su reti pubbliche.

\noindent Un'ulteriore sviluppo possibile è l'aggiunta di una videocamera a bordo del mezzo. Tale sensore può risultare molto vantaggioso sia ai fini della guida autonoma che di quella remota. L'integrazione di questo sensore infatti si rivelerebbe strategica per diverse ragioni. In primo luogo, essa fornisce all'operatore remoto una rappresentazione visiva dettagliata dell'ambiente circostante, complementando le informazioni provenienti dal Lidar e consentendo un controllo più preciso e intuitivo del mezzo. In secondo luogo, i dati acquisiti dalla telecamera possono essere utilizzati per sviluppare e affinare algoritmi di visione artificiale, fondamentali per l'implementazione di funzionalità avanzate di guida autonoma, quali il riconoscimento di oggetti e il tracciamento di percorsi. 

\noindent Un'ulteriore significativo miglioramento potrebbe essere ottenuto sostituendo l'attuale Lidar 2D con un sensore 3D. Questa evoluzione consentirebbe al sistema di guida autonoma di operare con una precisione notevolmente maggiore, riducendo al minimo gli errori derivanti da variazioni dinamiche dell'ambiente circostante. Inoltre, la rappresentazione tridimensionale dello spazio offrirebbe un quadro più completo e dettagliato per un eventuale operatore umano che dovesse intervenire in remoto.

\noindent Altro aspetto da considerare sarà l'implementazione di un sistema di platooning, per permettere al  server di poter guidare oltre che un unico mezzo anche una flotta, questo può rivelarsi molto vantaggioso se si prevede l'utilizzo di questa tecnologia, ad esempio, per il trasporto di merci.

\noindent Sarà infine cruciale per una valutazione completa, è necessario effettuare test in situazioni reali, simulando le potenziali limitazioni della rete e le complessità tipiche degli ambienti di utilizzo.

\section{Applicazioni pratiche}
Per concludere, si esamineranno le potenziali applicazioni concrete di questa innovazione.

\noindent In linea con le precedenti considerazioni, si prospetta l'opportunità di implementare flotte di veicoli semiautonomi in rete, appositamente progettate per il trasporto. L'adozione di rover autonomi in grado di operare in ambienti industriali di grandi dimensioni, movimentando ingenti quantitativi di materiali, costituirebbe un volano per l'innovazione tecnologica all'interno dell'azienda.

\noindent Altro utilizzo pratico si può avere nel caso di veicoli ad utilizzo personale. Si può infatti ipotizzare uno scenario in cui il guidatore non sia in grado di controllare il veicolo in caso di emergenza medica e che quindi si avvalga ad un servizio che preveda un operatore pronto a connettersi e che possa pilotare il mezzo a distanza. 
