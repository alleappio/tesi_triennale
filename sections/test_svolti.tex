\section{Test Svolti}
In questa sezione si elencano i vari test che sono stati svolti, i problemi riscontrati e le soluzioni trovate.

\subsection{Controllo}
Il primo test svolto è stato sulla parte che riguarda il controllo. Si è dovuto infatti verificare che i driver ed il nodo ROS forniti da AgileX per il rover modello Hunter e dedicati al controllo ed all'analisi del mezzo fossero utilizzabili senza necessarie modifiche o se invece fossero necessari aggiustamenti o addirittura una reimplementare.

\noindent Per svolgere questo test si è deciso di utilizzare un nodo ROS capace di ricevere dati da un joypad, elaborarli per poter ottenere un angolo di sterzo ed una velocità e pubblicarli sul topic ROS \textit{/drive\_parameters} sottoforma di messaggi ackermann, grazie a questo nodo infatti si poteva direttamente controllare se:

\begin{itemize}
  \item Il nodo ROS di casa AgileX comprendesse messaggi di tipo ackermann
  \item Il driver riuscisse a convertire efficacemente i messaggi ROS in messaggi CAN, così che il rover potesse eseguire i comandi impartiti.
\end{itemize}

\noindent I primi test non sono andati a buon fine, in quanto dopo un accurata analisi si è riscontrato che il nodo ROS di casa AgileX utilizza un diverso formato di messaggi per il controllo del mezzo.

\noindent La soluzione alla fine è stata quella di fare una piccola modifica al nodo ROS di casa AgileX per permettere a questo di interpretare la tipologia di messaggi corretta.

\subsection{Mappatura}
Il secondo test è stato quello di eseguire una mappatura bidimensionale di un intero ambiente.

\noindent Con mappatura bidimensionale si intende ricreare una vista dall'alto di un ambiente grazie all'utilizzo del sensore Lidar e dell'odometria del mezzo, sapendo infatti lo spostamento e la pointcloud rilevata dal sensore ed un algoritmo apposito è possibile ricreare questa mappa.

\noindent Nello specifico l'algoritmo utilizzato è chiamato SLAM (Simultaneous localization and mapping), e come dice il nome è un algoritmo utilizzato per la localizzazione e mappatura simultanea in un ambiente, l'algoritmo è implementato dal nodo \textbf{slam\_toolbox}, scaricabile gratuitamente.

\noindent Dopo un veloce setup e alcune prove si è dunque riusciti a ricreare una mappa fedele del primo piano dell'edificio di matematica del dipartimento di scienze fisiche, matematiche e informatiche di UniMoRe.

\subsection{Localizzazione}
Il terzo test è stato uno dei più importanti, questo riguarda la localizzazione.

\noindent Avendo una mappa dell'ambiente grazie ai test precedenti è stato infatti possibile provare ad avere una localizzazione all'interno dell'ambiente ricostruito, questo grazie al \textbf{particle\_filter}, algoritmo già discusso precedentemente.

\noindent I test in questo caso sono stati inconcludenti, è stato infatti notato che in alcuni punti dell'ambiente il particle filter non raggiungeva una precisione soddisfacente, questo a causa dell'ambiente stesso che risultava "feaureless", ovvero non vi erano differenze particolarmente apprezzabili tra due punti distinti della mappa.

\noindent Una soluzione a questo problema può essere sicuramente quella di utilizzare un Lidar più avanzato a tre dimensioni, in modo da apprezzare feature dell'ambiente che non sarebbero altrimenti rilevabili in due dimensioni, soluzione che sta venendo sperimentata al momento della stesura della presente tesi.

\subsection{Guida autonoma}
Uno degli ultimi test svolti è stato quello dello stack completo di guida autonoma. 
