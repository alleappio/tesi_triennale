\section{Test Svolti}
In questa sezione si elencano i vari test che sono stati svolti, i problemi riscontrati e le soluzioni trovate.

\subsection{Controllo}
Il primo test svolto è stato sulla parte che riguarda il controllo. Si è dovuto infatti verificare che i driver ed il nodo ROS forniti da AgileX per il rover modello Hunter e dedicati al controllo ed all'analisi del mezzo fossero utilizzabili senza necessarie modifiche o se invece fossero necessari aggiustamenti o addirittura una reimplementare.

\noindent Per svolgere questo test si è deciso di utilizzare un nodo ROS capace di ricevere dati da un joypad, elaborarli per poter ottenere un angolo di sterzo ed una velocità e pubblicarli sul topic ROS \textit{/drive\_parameters} sottoforma di messaggi ackermann, grazie a questo nodo infatti si poteva direttamente controllare se:

\begin{itemize}
  \item Il nodo ROS di casa AgileX comprendesse messaggi di tipo ackermann
  \item Il driver riuscisse a convertire efficacemente i messaggi ROS in messaggi CAN, così che il rover potesse eseguire i comandi impartiti.
\end{itemize}

\noindent I primi test non sono andati a buon fine, in quanto dopo un accurata analisi si è riscontrato che il nodo ROS di casa AgileX utilizza un diverso formato di messaggi per il controllo del mezzo.

\noindent La soluzione alla fine è stata quella di fare una piccola modifica al nodo ROS di casa AgileX per permettere a questo di interpretare la tipologia di messaggi corretta.
