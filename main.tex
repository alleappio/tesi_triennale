%\documentclass{article}
\documentclass[12pt,twoside,a4paper]{book}

\usepackage{graphicx} % Required for inserting images
\usepackage{listings}
\usepackage{xcolor}
\usepackage{forest} %p Per disegnare grafi
\usepackage{float}
\usepackage{natbib}
\usepackage{hyperref}
\usepackage{titlesec}
\usepackage[figurename=Figura]{caption}
\usepackage{etoolbox}
\usepackage[italian]{babel}
\usepackage{pdfpages}
%stacco tra paragrafi
\usepackage[skip=5pt plus1pt, indent=0pt]{parskip}
\usepackage{pgfplots}




\usepackage[left=3.5cm,top=3cm,right=2.5cm]{geometry}
\usepackage{fancyhdr}   %impaginazione
 \pagestyle{fancy}
\fancyhf{}
\fancyhead[LE,RO]{}
\fancyhead[RE,LO]{}
\fancyfoot[CE,CO]{}
\fancyfoot[LE,RO]{\thepage}
\renewcommand{\headrulewidth}{0.0pt}
\renewcommand{\footrulewidth}{0.0pt}

\linespread{1.3}\selectfont

\makeatletter  %per fare in stile FANCY tutte le pagine
\let\ps@plain=\ps@fancy
\makeatother






%interlinea
\renewcommand{\baselinestretch}{1} 

% Redefine \noindent to include vertical space before every occurrence

\titleformat{\chapter}[hang]{\normalfont\LARGE\bfseries}{Capitolo \thechapter}{0em}{\vspace{1em}\hrule\vspace{1em}}[\vspace{1em}\hrule\vspace{-0.5em}]
%\titleformat*{\subsection}{\normalfont\Large\bfseries}
%\titleformat*{\subsubsection}{\large\bfseries}
%\titleformat*{\paragraph}{\large\bfseries}
%\titleformat*{\subparagraph}{\large\bfseries}

\definecolor{pc_background}{rgb}{0.96, 0.96, 0.96} % Background light grey
\definecolor{pc_text}{rgb}{0.15, 0.15, 0.15}       % Dark text
\definecolor{pc_comment}{rgb}{0.45, 0.64, 0.82}    % Blueish comments
\definecolor{pc_keyword}{rgb}{0.86, 0.13, 0.13}    % Reddish keywords
\definecolor{pc_string}{rgb}{0.25, 0.50, 0.20}     % Greenish strings
\definecolor{pc_type}{rgb}{0.70, 0.13, 0.54}       % Magenta types
\definecolor{pc_function}{rgb}{0.35, 0.35, 0.75}   % Purple functions
\definecolor{pc_number}{rgb}{0.63, 0.37, 0.09}     % Orange numbers


\title{Università degli studi di Modena e Reggio Emilia\\
        Dipartimento di ingegneria "Enzo Ferrari"}

%\author{ALESSANDRO APPIO}
\date{2024}

\begin{document}
    \lstset{
      language=C++,                     % Linguaggio del codice
      basicstyle=\footnotesize\ttfamily\color{pc_text},   % Stile di base: monospazio e colore del testo
      backgroundcolor=\color{pc_background}, % Colore di sfondo
      commentstyle=\color{pc_comment},       % Colore dei commenti
      keywordstyle=\color{pc_keyword}\bfseries, % Colore e stile delle parole chiave
      stringstyle=\color{pc_string},         % Colore delle stringhe
      numberstyle=\tiny\color{pc_number},    % Colore dei numeri di riga
      identifierstyle=\color{pc_text},       % Colore delle variabili e identificatori
      frame=single,          % Cornice attorno al codice
      breaklines=true,
      numberstyle=\tiny\color{gray},    % Stile della numerazione
      numbers=left                     % Numerazione delle righe a sinistra
    }

    \frontmatter
    \includepdf[pages={1}]{frontespizio_tesi.pdf}
    \newpage
    \thispagestyle{empty}
    \cleardoublepage
    \newpage
    \null \vspace {\stretch {1}}
    \begin{flushright}
    \textit{Alla mia famiglia}
    \end{flushright}
    \vspace {\stretch {2}}\null
    \thispagestyle{empty}
    \newpage
    \thispagestyle{empty}
    \cleardoublepage
    \newpage
    \setcounter{page}{1}
    \chapter*{Abstract}

Col passare degli anni ci si approccia sempre di più a un mondo caratterizzato da una progressiva integrazione di sistemi automatici e connessi, che incide profondamente sulle modalità di interazione con l'ambiente e sulle abitudini di vita.

\noindent La presente tesi esplora i concetti di guida autonoma e guida remota, due paradigmi emergenti nel settore dei trasporti che promettono di rivoluzionare la mobilità. La guida autonoma è caratterizzata da veicoli in grado di percepire l'ambiente circostante, pianificare traiettorie e eseguire manovre senza intervento umano diretto, la guida remota invece prevede che le decisioni relative alla guida siano prese a distanza da un operatore umano o da un sistema intelligente.

\noindent L'analisi comparativa tra le due tecnologie evidenzia le loro specificità e potenzialità. La guida autonoma offre la prospettiva di veicoli completamente autonomi, in grado di operare in modo sicuro ed efficiente in una vasta gamma di scenari. La guida remota, da parte sua, consente di estendere le capacità di guida a veicoli non tradizionali o di operare in ambienti complessi.

\noindent Il presente lavoro si concentra sullo sviluppo di un sistema di guida caratterizzato da una dualità funzionale: la capacità di operare sia in modalità autonoma che in modalità remota. Nello specifico per lo sviluppo è stato utilizzato un rover di marca AgileX come piattaforma di sviluppo, su cui sono stati montati una GPGPU Nvidia AGX Xavier come computer di bordo ed un sensore Lidar per l'analisi ambientale. Per permettere lo sviluppo dello stack di guida remota ci si è avvalsi del protocollo MQTT, ampiamente utilizzato nelle tecnologie IOT e della suite di librerie ROS utilizzato per permettere una gestione modulare delle varie funzionalità del robot.

\noindent Il sistema, in conclusione si rivela funzionale, ma presenta alcune criticità relative alla latenza riscontrata nelle comunicazioni remote, con tempistiche misurate che si rivelano eccessive per l'ambito della guida remota, richiedendo dunque ulteriori approfondimenti ed eventuali miglioramenti nello stack realizzato.
    \newpage
    \input{sections/index.tex}
    \newpage

    \mainmatter
    
    \section{Introduzione}
\subsection{Guida remota e guida autonoma}
\subsection{Scopo della tesi}

    \newpage
    \section{Piattaforma di sviluppo}
In questa sezione si descrive come è composta e come è stata assemblata la piattaforma per lo sviluppo e il testing.
\subsection{Rover AgileX}
Il veicolo selezionato per lo sviluppo della presente tesi è un rover terrestre prodotto da AgileX, modello Hunter. Questo rover è dotato di un'interfaccia di controllo basata sul protocollo CAN (Controller Area Network), che consente una comunicazione efficiente e affidabile tra i diversi sistemi elettronici del veicolo. Il modello Hunter è stato scelto per le sue avanzate caratteristiche tecniche e per la sua versatilità, che lo rendono particolarmente adatto alle esigenze del progetto.

\noindent Oltre a fornire un'interfaccia per il controllo diretto, il veicolo è in grado di raccogliere e trasmettere una serie di dati diagnostici e operativi fondamentali per il monitoraggio e l'analisi delle sue prestazioni. Tra questi dati, un ruolo cruciale è ricoperto dall'odometria, che rappresenta la misura dello spostamento del veicolo basata sul movimento delle ruote. L'odometria è essenziale per la navigazione e la stima della posizione del rover, poiché permette di determinare il percorso seguito dal veicolo e la distanza percorsa. Questi dati, insieme ad altre informazioni sullo stato del veicolo, contribuiscono a garantire un controllo preciso e ad alimentare i sistemi di guida autonoma e remota previsti dal progetto.
\subsection{GPGPU e sensore lidar}
Un elemento cruciale per la realizzazione di questa tesi è stato l'identificazione e la selezione di un calcolatore embedded idoneo a gestire l'intera logica di controllo e la pianificazione, oltre alla scelta di un sensore in grado di fornire dati essenziali per la percezione dell'ambiente circostante.
\noindent Per il calcolatore embedded, è stata presa la decisione di impiegare una GPGPU (General Purpose Graphic Processing Unit), una scelta motivata dalla sua elevata capacità di elaborazione parallela, che risulta particolarmente vantaggiosa per eseguire complessi algoritmi di controllo e di pianificazione in tempo reale. La GPGPU selezionata opera con il sistema operativo Ubuntu 20.04, noto per la sua stabilità, ampia compatibilità con hardware di ultima generazione, e supporto per lo sviluppo di applicazioni avanzate, inclusi strumenti specifici per l'elaborazione grafica e la gestione di risorse computazionali.
\noindent Per quanto concerne il sensore, la scelta è ricaduta su un sensore Lidar (Light Detection and Ranging). Questo dispositivo sfrutta la tecnologia laser per determinare la distanza di vari punti nell'ambiente circostante, calcolando il tempo di ritorno dei raggi laser emessi. Il Lidar fornisce una mappa dettagliata della topografia dell'ambiente, consentendo al sistema di percezione di creare rappresentazioni tridimensionali accurate, fondamentali per il riconoscimento degli ostacoli, la navigazione e la pianificazione del percorso del veicolo. La combinazione di una GPGPU performante e un sensore Lidar avanzato rappresenta una solida base tecnologica per lo sviluppo di un sistema di guida autonoma e remota altamente efficiente.


    \newpage
    \chapter{Middleware di comunicazione} \label{middleware_di_comunicazione}
In questa sezione si discute dei principali componenti utilizzati per l'implementazione del middleware di comunicazione
\section{ROS}
In questa sezione si passa alla descrizione di ROS e del suo utilizzo.
\subsection{Nodi}
ROS o Robotic Operating System è un insieme di librerie e strumenti utili alla creazione di applicativi dedicati al controllo di robot.
Nello specifico ROS ci permette di creare unità di esecuzione o processi chiamati nodi. Un nodo ha la capacità di eseguire calcoli, interfacciarsi con periferiche o altro, ma la principale caratteristica di un nodo è la sua capacità di comunicare con gli altri nodi ROS in esecuzione.
\noindent Ciò ci permette di rappresentare con ogni nodo un modulo funzionale all'esecuzione della task del robot, per fare esempi pratici lo stack utilizzato per il controllo autonomo del rover è composto da diversi nodi, i principali sono: 
\noindent \begin{itemize}
  \item hunter\_ros2\_node: Gestisce la comunicazione tra i nodi ROS e l'interfaccia CAN del veicolo
  \item urg\_node: Comunica agli altri nodi la scan effettuata dal sensore lidar
  \item particle\_filter: calcola la localizzazione del mezzo a partire dalla mappa dell'ambiente e dalla scan del sensore
  \item telemetry\_node e control\_node: come descritto prima, gestiscono la comunicazione tra ROS ed MQTT 
\end{itemize}
Tutti questi nodi sono capaci di comunicano tra loro per scambiarsi informazioni utili all'esecuzione del robot.
\subsection{Comunicazione tra nodi}
Sorge però spontaneo domandarsi come questi nodi comunichino tra loro e come facciamo soprattutto a riconoscere di che tipo di informazione si tratti.
Una comunicazione ROS è formata da 3 elementi:
\begin{itemize}
  \item Topic: Il protocollo utilizzato da ROS è di tipo publish/subscribe, ciò vuol dire che durante l'esecuzione dei nodi si vanno a creare dei topic, ovvero stringhe che utilizzano come separatore il carattere '/' e che ci permettono di suddividere tutti i diversi dati da inviare. Un esempio sono le scan lidar che vengono pubblicate sul topic "/scan". Ogni nodo può decidere se fare la subscribe a quel nodo (ovvero ricevere tutti i dati inviati attraverso esso), fare delle publish (ovvero pubblicare dati su di esso) o se semplicemente ignorarlo.
  \item Message type: Una volta scelto un topic però si dovrà anche decidere quali informazioni saranno ammesse su questo, ROS fornisce diversi tipi di dato inviabile su un singolo topic. Un esempio è il tipo di dato utilizzato dal particle filter ovvero "Odometry messages", che descrivono la posizione (o meglio l'odometria) di un oggetto nello spazio ed è strutturanto nel seguente modo:

    \begin{forest}
      %for tree={grow'=90, circle, draw, l sep=5pt}
      for tree={draw}
      [Odometry
        [Header
          [timestamp]
        ]
        [Pose\_with\_covariance
          [Pose
            [Position
              [x]
              [y]
              [z]
            ]
            [Orientation
              [x]
              [y]
              [z]
              [w]
            ]
          ]
          [Covariance]
        ]
        [twist\_with\_covariance
          [Twist
            [Linear\_velocity
              [x]
              [y]
              [z]
            ]
            [Angular\_velocity
              [x]
              [y]
              [z]
            ]
          ]
          [Covariance]
        ]
      ]
    \end{forest}

  tutti i tipi di messaggio sono consultabili online sulla documentazione di ROS 
  \item Content: è il dato che dobbiamo inviare e che deve essere incapsulato nel tipo di dato fornitoci da ROS
\end{itemize}

\newpage
\section{MQTT}
In questa sezione si passa alla descrizione del protocollo di rete MQTT ed al perchè si è scelto di utilizzare questa tecnologia.
\subsection{Descrizione}
Il protocollo MQTT (Message Queuing Telemetry Transport) è un protocollo di rete di tipo publish-subscribe, progettato per la trasmissione di messaggi tra dispositivi in ambienti caratterizzati da connessioni di rete con larghezza di banda limitata, latenza elevata, o affidabilità intermittente.

\noindent Le principali caratteristiche del protocollo MQTT includono:

\begin{itemize}
  \item \textbf{Efficienza nella larghezza di banda}: MQTT è progettato per minimizzare l'overhead di rete, il che lo rende particolarmente adatto per applicazioni in cui la larghezza di banda è limitata o costosa.
    
  \item \textbf{Affidabilità e livelli di qualità del servizio (QoS)}: MQTT offre tre livelli di QoS, che consentono di bilanciare la necessità di affidabilità con le risorse disponibili. I livelli QoS vanno da "almeno una volta" a "esattamente una volta", garantendo diversi gradi di consegna del messaggio in base ai requisiti dell'applicazione.
    
  \item \textbf{Supporto}: per la persistenza delle sessioni: I client MQTT possono disconnettersi e riconnettersi senza perdere i messaggi inviati durante la disconnessione, grazie alla capacità del broker di mantenere lo stato delle sessioni e gestire i messaggi pendenti.

  \item \textbf{Sicurezza}: MQTT può essere configurato per utilizzare connessioni cifrate (SSL/TLS) e supporta l'autenticazione tramite username e password, garantendo la protezione dei dati scambiati e l'accesso controllato alle risorse.

  \item \textbf{Scalabilità}: La natura leggera e la flessibilità del modello publish-subscribe rendono MQTT altamente scalabile, consentendo di supportare un gran numero di dispositivi e applicazioni con un impatto minimo sulle risorse di rete.
\end{itemize}

\noindent Grazie a queste caratteristiche, MQTT è ampiamente utilizzato in una vasta gamma di applicazioni, tra cui la telemetria industriale, il monitoraggio ambientale, le smart cities, l'automazione domestica e i sistemi di gestione energetica, rappresentando una soluzione robusta ed efficiente per la comunicazione tra dispositivi eterogenei in contesti IoT.

\subsection{Infrastruttura}
\noindent MQTT opera secondo un'architettura client-server, dove i client (dispositivi o applicazioni) si connettono a un server (broker) centrale che gestisce la distribuzione dei messaggi. I client che desiderano inviare dati pubblicano messaggi su specifici argomenti (topics), mentre i client interessati a ricevere quei dati si iscrivono (subscribe) agli stessi argomenti. Il broker, che agisce come intermediario, si occupa di ricevere i messaggi pubblicati e di inoltrarli a tutti i client iscritti agli argomenti corrispondenti.

\subsection{Formattazione messaggi}
I messaggi scambiati tramite protocollo MQTT non sono altro che stringhe di testo. È quindi necessario utilizzare una formattazione per il testo che ci renda possibile distinguere i vari campi di un messaggio ROS (la cui struttura è illustrata nella sezione precedente) che vogliamo inoltrare. Per questa motivazione si è deciso di avvalersi del formato JSON che si presta bene a questo impiego.

\noindent Per fare un esempio di seguito si illustra come un messaggio di odometria (illustrato nella sezione precedente)si presenterà in forma di testo JSON:

\lstinputlisting[language=json]{samples/odometry.json}

\noindent Come si può vedere grazie a questa struttura è possibile rappresentare fedelmente i dati riportati dal messaggio ROS.

\subsection{Topic}
In MQTT, come in ROS, per dividere le varie tipologie di messaggi inoltrati si utilizzano i topic, questo in realtà è esattamente il motivo per cui si è deciso di utilizzare MQTT per il controllo remoto, infatti grazie a poche righe di codice è possibile ricavare i topic MQTT partendo da quelli ROS.

\noindent Conoscendo infatti la struttura dei topic ROS utilizzati e quella dei topic MQTT ci basterà eseguire il seguente codice per passare dall'uno all'altro:

\lstinputlisting{samples/ros_topic_to_mqtt_topic.cpp}

    %\section{ROS}
In questa sezione si passa alla descrizione di ROS e del suo utilizzo.
\subsection{Nodi}
ROS o Robotic Operating System è un insieme di librerie e strumenti utili alla creazione di applicativi dedicati al controllo di robot.
Nello specifico ROS ci permette di creare unità di esecuzione o processi chiamati nodi. Un nodo ha la capacità di eseguire calcoli, interfacciarsi con periferiche o altro, ma la principale caratteristica di un nodo è la sua capacità di comunicare con gli altri nodi ROS in esecuzione.
\noindent Ciò ci permette di rappresentare con ogni nodo un modulo funzionale all'esecuzione della task del robot, per fare esempi pratici lo stack utilizzato per il controllo autonomo del rover è composto da diversi nodi, i principali sono: 
\noindent \begin{itemize}
  \item hunter\_ros2\_node: Gestisce la comunicazione tra i nodi ROS e l'interfaccia CAN del veicolo
  \item urg\_node: Comunica agli altri nodi la scan effettuata dal sensore lidar
  \item particle\_filter: calcola la localizzazione del mezzo a partire dalla mappa dell'ambiente e dalla scan del sensore
  \item telemetry\_node e control\_node: come descritto prima, gestiscono la comunicazione tra ROS ed MQTT 
\end{itemize}
Tutti questi nodi sono capaci di comunicano tra loro per scambiarsi informazioni utili all'esecuzione del robot.
\subsection{Comunicazione tra nodi}
Sorge però spontaneo domandarsi come questi nodi comunichino tra loro e come facciamo soprattutto a riconoscere di che tipo di informazione si tratti.
Una comunicazione ROS è formata da 3 elementi:
\begin{itemize}
  \item Topic: Il protocollo utilizzato da ROS è di tipo publish/subscribe, ciò vuol dire che durante l'esecuzione dei nodi si vanno a creare dei topic, ovvero stringhe che utilizzano come separatore il carattere '/' e che ci permettono di suddividere tutti i diversi dati da inviare. Un esempio sono le scan lidar che vengono pubblicate sul topic "/scan". Ogni nodo può decidere se fare la subscribe a quel nodo (ovvero ricevere tutti i dati inviati attraverso esso), fare delle publish (ovvero pubblicare dati su di esso) o se semplicemente ignorarlo.
  \item Message type: Una volta scelto un topic però si dovrà anche decidere quali informazioni saranno ammesse su questo, ROS fornisce diversi tipi di dato inviabile su un singolo topic. Un esempio è il tipo di dato utilizzato dal particle filter ovvero "Odometry messages", che descrivono la posizione (o meglio l'odometria) di un oggetto nello spazio ed è strutturanto nel seguente modo:

    \begin{forest}
      %for tree={grow'=90, circle, draw, l sep=5pt}
      for tree={draw}
      [Odometry
        [Header
          [timestamp]
        ]
        [Pose\_with\_covariance
          [Pose
            [Position
              [x]
              [y]
              [z]
            ]
            [Orientation
              [x]
              [y]
              [z]
              [w]
            ]
          ]
          [Covariance]
        ]
        [twist\_with\_covariance
          [Twist
            [Linear\_velocity
              [x]
              [y]
              [z]
            ]
            [Angular\_velocity
              [x]
              [y]
              [z]
            ]
          ]
          [Covariance]
        ]
      ]
    \end{forest}

  tutti i tipi di messaggio sono consultabili online sulla documentazione di ROS 
  \item Content: è il dato che dobbiamo inviare e che deve essere incapsulato nel tipo di dato fornitoci da ROS
\end{itemize}

    %\newpage
    %\section{MQTT}
In questa sezione si passa alla descrizione del protocollo di rete MQTT ed al perchè si è scelto di utilizzare questa tecnologia.
\subsection{Descrizione}
Il protocollo MQTT (Message Queuing Telemetry Transport) è un protocollo di rete di tipo publish-subscribe, progettato per la trasmissione di messaggi tra dispositivi in ambienti caratterizzati da connessioni di rete con larghezza di banda limitata, latenza elevata, o affidabilità intermittente.

\noindent Le principali caratteristiche del protocollo MQTT includono:

\begin{itemize}
  \item \textbf{Efficienza nella larghezza di banda}: MQTT è progettato per minimizzare l'overhead di rete, il che lo rende particolarmente adatto per applicazioni in cui la larghezza di banda è limitata o costosa.
    
  \item \textbf{Affidabilità e livelli di qualità del servizio (QoS)}: MQTT offre tre livelli di QoS, che consentono di bilanciare la necessità di affidabilità con le risorse disponibili. I livelli QoS vanno da "almeno una volta" a "esattamente una volta", garantendo diversi gradi di consegna del messaggio in base ai requisiti dell'applicazione.
    
  \item \textbf{Supporto}: per la persistenza delle sessioni: I client MQTT possono disconnettersi e riconnettersi senza perdere i messaggi inviati durante la disconnessione, grazie alla capacità del broker di mantenere lo stato delle sessioni e gestire i messaggi pendenti.

  \item \textbf{Sicurezza}: MQTT può essere configurato per utilizzare connessioni cifrate (SSL/TLS) e supporta l'autenticazione tramite username e password, garantendo la protezione dei dati scambiati e l'accesso controllato alle risorse.

  \item \textbf{Scalabilità}: La natura leggera e la flessibilità del modello publish-subscribe rendono MQTT altamente scalabile, consentendo di supportare un gran numero di dispositivi e applicazioni con un impatto minimo sulle risorse di rete.
\end{itemize}

\noindent Grazie a queste caratteristiche, MQTT è ampiamente utilizzato in una vasta gamma di applicazioni, tra cui la telemetria industriale, il monitoraggio ambientale, le smart cities, l'automazione domestica e i sistemi di gestione energetica, rappresentando una soluzione robusta ed efficiente per la comunicazione tra dispositivi eterogenei in contesti IoT.

\subsection{Infrastruttura}
\noindent MQTT opera secondo un'architettura client-server, dove i client (dispositivi o applicazioni) si connettono a un server (broker) centrale che gestisce la distribuzione dei messaggi. I client che desiderano inviare dati pubblicano messaggi su specifici argomenti (topics), mentre i client interessati a ricevere quei dati si iscrivono (subscribe) agli stessi argomenti. Il broker, che agisce come intermediario, si occupa di ricevere i messaggi pubblicati e di inoltrarli a tutti i client iscritti agli argomenti corrispondenti.

\subsection{Formattazione messaggi}
I messaggi scambiati tramite protocollo MQTT non sono altro che stringhe di testo. È quindi necessario utilizzare una formattazione per il testo che ci renda possibile distinguere i vari campi di un messaggio ROS (la cui struttura è illustrata nella sezione precedente) che vogliamo inoltrare. Per questa motivazione si è deciso di avvalersi del formato JSON che si presta bene a questo impiego.

\noindent Per fare un esempio di seguito si illustra come un messaggio di odometria (illustrato nella sezione precedente)si presenterà in forma di testo JSON:

\lstinputlisting[language=json]{samples/odometry.json}

\noindent Come si può vedere grazie a questa struttura è possibile rappresentare fedelmente i dati riportati dal messaggio ROS.

\subsection{Topic}
In MQTT, come in ROS, per dividere le varie tipologie di messaggi inoltrati si utilizzano i topic, questo in realtà è esattamente il motivo per cui si è deciso di utilizzare MQTT per il controllo remoto, infatti grazie a poche righe di codice è possibile ricavare i topic MQTT partendo da quelli ROS.

\noindent Conoscendo infatti la struttura dei topic ROS utilizzati e quella dei topic MQTT ci basterà eseguire il seguente codice per passare dall'uno all'altro:

\lstinputlisting{samples/ros_topic_to_mqtt_topic.cpp}

    \newpage
    \chapter{Scelte progettuali}
Una delle peculiarità di ROS è che non limita la comunicazione tra i nodi al solo device che esegue l'istanza, ma può distribuire i pacchetti pubblicati dai nodi anche tramite rete, avvalendosi del protocollo UDP. 
Questo significa che per implementare la comunicazione necessaria al funzionamento della guida remota sarebbe stato suficiente sfruttare questa peculiarità di ROS. Si riporta nelle seguenti sezioni le motivazioni per le quali si è deciso di utilizzare invece lo stack MQTT.

\section{Affidabilità} \label{scelte_progettuali_affidabilità}
Come descritto precedentemente, il framework ROS prevede l'esclusivo utilizzo del protocollo UDP per l'invio di messaggi in rete. Il protocollo UDP è un protocollo a livello trasporto molto veloce e versatile, e viene soprattutto utilizzato in applicazioni orientate a minimizzare l'overhead e la latenza. Tuttavia, questa scelta comporta alcuni compromessi. 

\noindent Essendo un protocollo connectionless, il protocollo UDP non garantisce la consegna dei pacchetti né il loro ordine di arrivo. Inoltre, non fornisce meccanismi di controllo dell'errore come la ritrasmissione automatica dei pacchetti persi. Di conseguenza, in ambienti con elevata interferenza o congestione di rete, si possono verificare perdite di dati e una degradazione della qualità del servizio.

\noindent Questa caratteristica del protocollo non ne permette un'utilizzo affidabile in casi critici come quello di studio, in quanto una perdita di pacchetti potrebbe molto facilmente scalare in potenziali danni al veicolo, persone od oggetti terzi.

\noindent Al contrario, il protocollo MQTT si avvale del protocollo TCP a livello di trasporto. Questa scelta progettuale offre una serie di vantaggi che lo rendono particolarmente adatto per applicazioni IoT e di messaggistica in generale.

\noindent Il protocollo TCP garantisce infatti la consegna ordinata e affidabile dei messaggi, riducendo al minimo il rischio di perdite di dati. Questa caratteristica è fondamentale in scenari come quello studiato, dove l'integrità dei dati è cruciale. Questa garanzia ci viene fornita da diverse tecniche che TCP implementa come i meccanismi di acknowledge e timeout dei pacchetti. 

\begin{figure}[H]
  \centering
  \includegraphics[width=1\textwidth]{figures/tcp_vs_udp_geeksforgeeks.png}
  \caption{Comparazione graficata tra i 2 protocolli. Immagine tratta da \cite{TCPvsUDP_geeksforgeeks}}
  \label{tcp_v_udp}
\end{figure}

\noindent Come mostrato in figura \ref{tcp_v_udp}, il protocollo TCP implementa infatti meccanismi di controllo del flusso che evitano la congestione della rete, garantendo una comunicazione efficiente anche in condizioni di carico elevato.

\noindent Tuttavia, il protocollo TCP è un protocollo sensibilmente più "pesante" e lento rispetto al protocollo UDP. Nonostante questo, la latenza introdotta dall'uso di questo protocollo risulta accettabile per il funzionamento del sistema e si è deciso di sfruttare il protocollo MQTT per l'affidabilità nella trasmissione. Questa scelta è stata dettata dalla necessità di garantire la consegna dei messaggi in modo sicuro e ordinato, anche in condizioni di rete instabili.

\section{Sicurezza}
\noindent Nell'ambito dell'analisi di scenari di rete pubblica come il nostro, l'aspetto della sicurezza riveste un'importanza cruciale. In questo contesto, l'adozione di ROS per la guida remota potrebbe rivelarsi una scelta non ottimale.

\noindent ROS, nella sua configurazione standard, non prevede l'implementazione di meccanismi di sicurezza a livello di pacchetto. Di conseguenza, i dati scambiati tra i nodi della rete vengono trasmessi in chiaro, rendendoli potenzialmente accessibili a qualsiasi entità connessa alla rete. Questa vulnerabilità espone i sistemi a rischi di intercettazione, manipolazione o alterazione dei dati, con potenziali conseguenze negative sulla privacy e sulla sicurezza operativa.

\noindent Al contrario, MQTT supporta nativamente lo strato di trasporto sicuro SSL/TLS, che fornisce una crittografia end-to-end dei dati scambiati tra i dispositivi. Questo significa che anche in caso di intercettazione delle comunicazioni, i dati rimarranno incomprensibili agli esterni.

\section{Struttura}
Nell'ambito della guida remota, la resilienza del sistema è un fattore critico. Un'infrastruttura deve essere in grado di operare in modo affidabile anche in condizioni avverse, garantendo la continuità del servizio e la sicurezza dei dati. In questo contesto, MQTT dimostra una maggiore resilienza rispetto a ROS. 

\noindent MQTT è progettato per operare in ambienti distribuiti, con molti dispositivi connessi a un broker centrale. Questa architettura rende il sistema più resistente a guasti locali, poiché la perdita di un singolo dispositivo o di una connessione non compromette necessariamente l'intero sistema.

\noindent Al contrario, ROS spesso si basa su un unico organo, chiamato \textit{ROS Master}, che coordina tutte le comunicazioni. La perdita o l'irraggiungibilità di questo organo può causare il blocco dell'intero sistema.

\begin{figure}[H]
  \centering
  \includegraphics[width=1\textwidth]{figures/ros_master.png}
  \caption{Struttura del sistema ROS}
  \label{struttura_ros}
\end{figure}

\section{Controllo del QoS}
Un ulteriore punto da considerare è quella della gestione della Quality Of Service  (QoS). Il concetto di QoS indica il livello di garanzia che una rete offre per quanto riguarda la consegna dei dati. Una bassa QoS indica che i pacchetti non vengono consegnati correttamente o che addirittura non è certa la ricezione di questi messaggi.  

\noindent Entrambi i protocolli offrono meccanismi di gestione della QoS, nelle seguenti modalità.

\noindent MQTT offre tre livelli di gestione (0, 1, 2) che consentono di adattare il livello di affidabilità della rete. Tutti e tre i livelli si basano sul come e quante volte il messaggio deve essere riinviato perchè questo possa essere considerato ricevuto. 

\begin{itemize}
  \item QoS 0 (At most once): il messaggio viene consegnato al massimo una volta. Non c'è alcuna garanzia di consegna, e il messaggio potrebbe perdersi. È il livello più veloce ma meno affidabile
  \item QoS 1 (At least once): il messaggio viene consegnato almeno una volta. Il broker invia un messaggio di conferma al publisher, e se non riceve conferma, reinvia il messaggio. Questo livello garantisce che il messaggio arrivi, ma potrebbe arrivare più di una volta
  \item QoS 2 (Exactly once): il messaggio viene consegnato esattamente una volta. Il broker invia un messaggio di conferma al publisher, e solo dopo aver ricevuto questa conferma, considera il messaggio consegnato. Questo è il livello più affidabile ma anche il più lento
\end{itemize}

\noindent Diversamente, ROS non offre un meccanismo di QoS così definito e flessibile come MQTT. La gestione della qualità del servizio in ROS è più legata alla configurazione dei nodi e dei topic che, spesso, richiede un'implementazione personalizzata per garantire un livello di affidabilità specifico. 

Le possibili configurazioni che è possibile fare sui nodi ROS sono:
\begin{itemize}
  \item History: definisce la quantità di dati che possono essere memorizzati in un buffer prima che vengano scartati.
  \item Depth: indica la dimensione massima del buffer.
  \item Reliability: determina il livello di affidabilità della comunicazione (best-effort, reliable, etc.).
  \item Durability: specifica se i dati devono essere persistenti anche se i nodi non sono connessi.
  \item Liveliness: definisce come spesso un nodo deve dimostrare di essere attivo.
\end{itemize}


    \newpage
    \section{Funzionamento}
Nella seguente sezione si descrive il funzionamento generale dello stack di guida remota e come un server si può interfacciare con il veicolo per svolgere tali operazioni
\subsection{Informazioni scambiate tra veicolo e server}
Per adempire allo scopo di guida remota sono necessarie due figure distinte: Il veicolo ed un server. Il veicolo è stato descritto nella sezione precedente, il server invece è quella figura del sistema con cui l'umano si interfaccia e che fornisce sia i dati prelevati dal veicolo sia un metodo per il controllo del veicolo.
Per garantire il funzionamento della struttura è quindi necessario lo scambio di informazioni tra le due parti. Queste informazioni sono:
- I messsaggi di controllo per i motori
- La pointcloud fornita dal lidar
- l'odometria calcolata dal veicolo
Le informazioni tra il server ed il veicolo sono scambiate tramite protocollo di rete MQTT, un protocollo studiato per il mondo dell'IOT e per tali applicazioni.
Sia a bordo del veicolo che sul server è istanziata un versione di ROS (Robotic Operative System) ovvero un software utilizzato per creare diversi nodi (o processi paralleli) e per permettere lo scambio di informazioni tra essi (Tramite topic e messaggi).
\subsection{Funzionamento lato veicolo}
A bordo del mezzo sono istanziati due nodi ROS: uno incaricato di gestire l'invio dei dati del sensore e dell'odometria, ed uno incaricato di ricevere i messaggi di controllo. Per comodità li chiameremo rispettivamente \textit{telemetry node} e \textit{control node}. 
Come descritto prima il \textit{telemetry node} si incarica d ricevere messaggi da ROS contenenti la pointcloud del lidar e l'odometria del mezzo, per poi formattare tali messaggi come stringhe JSON ed inviare questi messaggi tramite protocollo MQTT al server in due topic dedicati.
Per quanto riguarda il \textit{control node} invece, questo si incarica di ricevere messaggi dal server che riguardano il controllo del mezzo, una volta ricevuti i messaggi come stringa JSON questi vengono convertiti in messaggi ROS e vengono inviati ai driver del rovere che permettono poi il controllo tramite CAN del veicolo.
\subsection{Funzionamento lato server}

    \newpage
    \section{Validazione sperimentale}
In questa sezione si elencano i vari test che sono stati svolti, i problemi riscontrati e le soluzioni trovate.

\subsection{Controllo}
La prima fase sperimentale è stata dedicata alla verifica funzionale dei driver e del nodo ROS forniti da AgileX per il rover modello Hunter. L'obiettivo primario era determinare se tali componenti, specificamente progettati per il controllo e l'analisi del veicolo, potessero essere integrati nel sistema senza richiedere modifiche sostanziali o se, al contrario, fosse necessario apportare adattamenti o addirittura una completa riscrittura.

\noindent Per condurre questa valutazione, è stato sviluppato un nodo ROS dedicato alla ricezione dei dati da un joystick. Questi dati, dopo un'elaborazione preliminare, venivano convertiti in un angolo di sterzo e una velocità che, successivamente, venivano pubblicati sul topic ROS \textit{/drive\_parameters} sotto forma di messaggi di tipo Ackermann. Tale configurazione consentiva di verificare direttamente:

\begin{itemize}
  \item Compatibilità del nodo ROS AgileX: se il nodo ROS fornito dal produttore supportasse il formato dei messaggi Ackermann, comunemente utilizzato per il controllo di veicoli mobili
  \item Efficacia del driver: se il driver fosse in grado di tradurre correttamente i messaggi ROS in messaggi CAN, permettendo così al rover di eseguire i comandi impartiti
\end{itemize}

\noindent I primi test non sono andati a buon fine in quanto, dopo un'accurata analisi, si è riscontrato che il nodo ROS di casa AgileX utilizza un diverso formato di messaggi per il controllo del mezzo, interrompendo così lo stack.

\noindent La soluzione che si è deciso di adottare è stata di introdurre una modifica puntuale al nodo ROS AgileX. Tale modifica ha consentito al nodo di interpretare correttamente il tipo di messaggi previsto, ripristinando così la compatibilità con il resto del sistema

\subsection{Mappatura}
Il secondo test è stato quello di eseguire una mappatura bidimensionale di un intero ambiente.

\noindent Con mappatura bidimensionale si intende ricreare una vista dall'alto di un ambiente grazie all'utilizzo del sensore Lidar e dell'odometria del mezzo. Conoscendo infatti lo spostamento e la pointcloud rilevata dal sensore, tramite un apposito algoritmo è possibile ricreare questa mappa.

\noindent Per realizzare questa operazione, è stato impiegato un algoritmo di Simultaneous Localization and Mapping (SLAM). Come suggerisce il nome, lo SLAM è una tecnica che consente di localizzare un robot all'interno di un ambiente sconosciuto e, contemporaneamente, di costruire una mappa di tale ambiente. In questo caso specifico, l'algoritmo SLAM è stato implementato nel nodo ROS \textbf{slam\_toolbox}, un pacchetto software open-source ampiamente utilizzato nella comunità robotica.

\noindent Dopo una fase di configurazione iniziale e alcune prove preliminari, è stato possibile generare una mappa bidimensionale accurata del primo piano dell'edificio di matematica del Dipartimento di Scienze Fisiche, Matematiche e Informatiche dell'Università di Modena e Reggio Emilia (UNIMORE). La mappa ottenuta rappresenta una fedele rappresentazione planimetrica dell'ambiente, evidenziando con precisione gli ostacoli presenti e le caratteristiche geometriche delle pareti

\begin{figure}[H]
  \centering
  \includegraphics[width=0.5\textwidth]{figures/franco_map.png}
  \caption{Mappa ottenuta tramite Lidar del primo piano dell'edificio Matematica}
  \label{franco_map}
\end{figure}


\noindent La mappa ottenuta grazie a questi test è riportata in figura \ref{franco_map}.
\subsection{Localizzazione}
Il terzo test è stato uno dei più importanti, e riguarda la localizzazione.

\noindent Avendo una mappa dell'ambiente grazie ai test precedenti è stato infatti possibile svolgere test volti al calcolo preciso della propria posizione all'interno di un ambiente conosciuto. Questi test sono stati svolti grazie al nodo ROS \textbf{particle\_filter} che implementa l'algoritmo di localizzazione già discusso alla sezione \ref{funzionamento_autonomo_perception}.

\noindent I test di localizzazione condotti hanno evidenziato alcune criticità legate alla natura dell'ambiente di prova. In particolare, sono state riscontrate difficoltà nel raggiungimento di una stima accurata della posizione in determinate aree dell'ambiente, caratterizzate da una scarsità di elementi distintivi (featureless). Tale condizione ha limitato la capacità del filtro a particelle di discriminare tra posizioni potenzialmente simili, compromettendo così la precisione della localizzazione.

\noindent Una soluzione a questo problema può essere sicuramente quella di utilizzare un Lidar più avanzato a tre dimensioni, in modo da apprezzare feature dell'ambiente che non sarebbero altrimenti rilevabili in due dimensioni, soluzione che sta venendo sperimentata al momento della stesura della presente tesi.

\subsection{Guida autonoma}
La penultima sperimentazione ha avuto come obiettivo la verifica delle prestazioni complessive del sistema di navigazione autonoma. A partire dalla mappa dell'ambiente generata precedentemente e dai parametri di controllo ottimizzati, è stato pianificato un percorso di riferimento. Successivamente, il robot è stato incaricato di seguire il percorso predefinito, utilizzando l'algoritmo di localizzazione per stimare la propria posizione in tempo reale e adattare la traiettoria in base alle informazioni sensoriali acquisite.

\noindent Sebbene i risultati ottenuti siano stati generalmente positivi, si sono manifestati alcuni problemi di deviazione dalla traiettoria pianificata, attribuibili alle incertezze nella stima di posizione e particolarmente evidenti nelle aree dell'ambiente prive di elementi distintivi.

\subsection{Guida remota}
In conclusione, sono stati svolti test per quanto concerne la guida remota. Nello specifico si è utilizzato un gamepad connesso ad un computer portatile che andasse a comunicare con il nodo ROS \textit{mqtt\_server\_node} per l'impartizione e per l'invio dei comandi tramite rete. Contemporaneamente, sul veicolo venivano eseguiti i nodi \textbf{mqtt\_control\_node} per la ricezione dei comandi e \textbf{hunter\_ros2\_node} per inviare i comandi all'interfaccia CAN del veicolo.

\noindent Per la comunicazione MQTT è stato utilizzato un broker hostato su rete pubblica con cui sia il veicolo che il portatile (che nell'esperimento rappresentava il server) andavano a comunicare. 

\noindent Le verifiche effettuate hanno confermato il corretto funzionamento degli applicativi, sebbene abbiano messo in luce alcune inefficienze. Tra queste, la latenza nella comunicazione tra veicolo e stazione di controllo è risultata particolarmente critica. Il tempo di risposta, misurato tra l'invio di un comando dal gamepad e la corrispondente azione del veicolo, ha mostrato valori elevati, nell'ordine dei secondi, rendendo l'applicazione non idonea a scenari che richiedono tempi di reazione rapidi come la guida remota.

\noindent Gli stessi tempi di latenza di questo test sono stati anche riscontrati nelle prove che riguardavano la telemetria: l'invio di dati da parte del mezzo che riguardano i sensori a bordo al server.

    \newpage
    \section{Sviluppi futuri e conclusioni}
Per concludere, si va ad analizzare quali complicazioni sono state riscontrate durante lo sviluppo, quali falle rimangono ed eventuali soluzioni e sviluppi futuri e applicazioni pratiche della tesi.

\subsection{Problemi ed eventuali soluzioni}
Il primo dubbio riguardo il progetto descritto in questa tesi è sicuramente quello della latenza di rete: Applicazioni come quella della guida autonoma vengono chiamate real time, ciò vuol dire che l'esecuzione di ogni singolo pezzo dello stack deve eseguire in tempi stretti e che anche nel caso peggiore l'esecuzione non può superare una certa quantità di tempo, questa quantità viene chiamata deadline.  

\noindent È quindi spontaneo porsi un dubbio, ovvero quanto l'utilizzo della rete complichi il dover far rispettare le tempistiche. Includere la rete in applicazioni real time infatti diventa rischioso, si pensa subito al caso in cui la connessione sia molto scarsa o addirittur assente, quali complicazioni questo può portare. Anche nel caso di una situazione ottimale però bisogna sempre considerare quanto l'inclusione di un protocollo di rete (sepur leggero come nel caso di MQTT) porti dell'overhead e di conseguenza delle latenze nell'esecuzione.

\noindent Possiamo ovviare a questi problemi (seppur limitatamente) grazie al meccanismo dei livelli di Quality Of Service che il protcollo ci fornisce, un livello basso di Quality Of Service infatti farà fare meno controlli al protocollo e di conseguenza rimuoverà un certo overhead. Anche la scelta del protocollo a livello transport influenza le latenze, se usare quindi UDP o TCP dato che notoriamente il protocollo UDP riduce di molto la complessità, a scapito però sempre della qualità del servizio.

\subsection{Sviluppi futuri}
Il progetto ai fini della tesi può considerarsi concluso, rimangono però eventuali implementazioni e test che potranno essere integrati in futuro.

\noindent La prima cosa che viene in mente è l'implementazione di uno stack di sicurezza informatica che permetta l'invio dei dati del mezzo su rete pubblica completamente criptati ed oscurati ad un possibile attaccante, implementazione necessaria se si prevede di utilizzare wurdts tecnologia in casi reali.

\noindent Un'ulteriore sviluppo possibile è l'aggiunta di una videocamera a bordo del mezzo. tale sensore può risultare molto vantaggioso sia ai fini della guida autonoma che di quella remota. Una telecamera infatti risulterebbe utile sia ad un operatore che si avvale della guida remota per poter osservare l'ambiente circostante con più chiarezza, rispetto a quanto il singolo sensore Lidar permetta, potrebbe inoltre risultare interessante per lo sviluppo di algoritmi di computer vision di cui la guida autonoma si avvarrebbe. 

\noindent Altro aspetto da considerare sarà l'implementazione di un sistema di platooning, per permettere al  server di poter guidare oltre che un unico mezzo anche una flotta, questo può rivelarsi molto vantaggioso se si prevede l'utilizzo di questa tecnologia , ad esempio, per il trasporto di merci.

\noindent Infine sarà necessario svolgere test in condizioni reali, condizioni in cui l'affidabilità alla rete sia limità o con ambienti molto complessi.

\subsection{Applicazioni pratiche}
Per concludere si procede ad elencare quali possono essere delle eventuali applicazioni pratiche di questa tecnologia.

\noindent Come descritto prima, un utilizzo potrebbe essere quello della creazione di una flotta di veicoli semi-autonomi connessi a scopi di trasporto, avere una flotta infatti di rover capaci di trasportare all'interno di ambienti lavorativi grandi quantità di materiale o semi-lavorati, aiuterebbe con lo sviluppo tecnologico di un'impresa.

\noindent Altro utilizzo pratico si può avere nel caso di veicoli ad utilizzo personale. Si può infatti ipotizzare uno scenario in cui il guidatore non sia in grado di controllare il veicolo in caso di emergena medica e che quindi si avvalga ad un servizio che preveda un operatore pronto a connettersi che possa pilotare il mezzo a distanza, o addirittura ad un servizio di guida autonoma che possa guidare il veicolo fino all'ospedale più vicino. 

    \newpage
    \chapter{Ringraziamenti}
Vorrei dedicare quest'ultimo capitolo per porre i giusti ringraziamenti a tutte le persone che mi hanno supportato nello svolgimento di questo lavoro e durante tutto il percorso di studi che ho intrapreso fino ad ora.

\noindent Vorrei in primis ringraziare il mio relatore, Prof. Paolo Burgio e tutti i miei colleghi di laboratorio per avermi dato la possibilità di realizzare il presente lavoro e per avermi seguito durante tutto lo svolgimento.

\noindent La mia famiglia, che più di chiunque altro mi ha non solo insegnato l'importanza dell'istruzione e del percorso che ho intrapreso e che sto intraprendendo ma soprattutto di avermi supportato e ascoltato anche nei momenti più difficili, in particolare mia madre Vanessa Toni.

\noindent I miei amici, le mie rocce, che mi hanno sostenuto ed aiutato in ogni momento, sia nel bene che nel male.

\noindent Ringrazio particolarmente Davide Paltrinieri per avermi accompagnato ininterrotamente da parecchi anni oramai e per essere sempre stato un punto di riferimento nella mia vita.

\noindent Grazie infine a tutte le persone che mi hanno sempre supportato, aiutato e accompagnato in questo percorso. 

    
    \newpage
    
    \nocite{*}
    \bibliographystyle{plain}
    \bibliography{bibliography/bibl}{}
    
\end{document}
